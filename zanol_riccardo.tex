\documentclass{article}

\usepackage[T1]{fontenc}
\usepackage[utf8]{inputenc}
\usepackage[italian]{babel}

\usepackage{amsmath}
\usepackage{siunitx}
\usepackage{hyperref}
\usepackage{listings}
\usepackage{color}
\usepackage{textcomp}
\usepackage{graphicx}

\definecolor{matlabgreen}{RGB}{28,172,0}
\definecolor{matlablilas}{RGB}{170,55,241}

\renewcommand{\lstlistingname}{Codice}

\renewcommand{\thesubsection}{\arabic{subsection}}
\makeatletter
\def\@seccntformat#1{\@ifundefined{#1@cntformat}%
   {\csname the#1\endcsname\quad}%       default
   {\csname #1@cntformat\endcsname}}%    enable individual control
\newcommand\section@cntformat{}
\makeatother

\newcommand{\includecode}[1]{\lstinputlisting[caption={\ttfamily #1},
    label={lst:#1}]{#1}}
\newcommand{\fig}[1]{Fig.~\ref{#1}}
\newcommand{\cod}[1]{Cod.~\ref{#1}}
\newcommand{\tab}[1]{Tab.~\ref{#1}}
\newcommand{\eqn}[1]{(\ref{#1})}
\newcommand{\inlcd}[1]{\lstinline[basicstyle=\ttfamily,keywordstyle={}]{#1}}

\title{Progetto di Elaborazione Numerica dei Segnali}
\author{Riccardo Zanol - mat. 1127207}
\date{\today}

\begin{document}
\lstset{
  language=Matlab,
  basicstyle={\ttfamily \footnotesize},
  breaklines=true,
  morekeywords={true,false,warning,xlim},
  keywordstyle=\color{blue},
  stringstyle=\color{matlablilas},
  commentstyle={\color{matlabgreen} \itshape},
  numberstyle={\ttfamily \tiny},
  frame=leftline,
  showstringspaces=false,
  numbers=left,
  upquote=true,
}
\sisetup{
  list-final-separator={ e }
}
\maketitle

Lo script Matlab riportato nel \cod{lst:progetto.m}, dopo aver letto
il file audio e definito alcuni parametri che saranno usati in
seguito, elabora il segnale $y(nT)$ nelle righe da 17 a 46 cercando il
rumore di ampiezza maggiore, stimandone ampiezza e instante d'inizio
una volta nota la frequenza, applicando un filtro notch per eliminare
il disturbo e ripetendo il ciclo \inlcd{while} finchè non viene più
rilevato alcun rumore sinusoidale.

\subsection{Frequenza}
\label{sec:freq}
La DFT di ognuno dei rumori $w_i(nT)$ è per definizione
\begin{equation}
  W_i(kF) = T \sum_{n=0}^{N-1}A_i\cos(2\pi f_i nT)e^{-j2\pi kF nT},
\end{equation}
che, applicando la formula di Eulero, diventa
\begin{equation}
  W_i(kF) = \frac{A_iT}{2} \left(
  \sum_{n=0}^{N-1}e^{-j2\pi (kF-f_i) nT} +
  \sum_{n=0}^{N-1}e^{-j2\pi (kF+f_i) nT}
  \right)
\end{equation}
e, se la frequenza del rumore $f_i$ è un multiplo del quanto
frequenziale $F$, si riduce a una coppia di impulsi:
\begin{equation}
  W_i(kF) = \frac{A_iTN}{2} \left( \delta(kF - f_i) + \delta(kF + f_i) \right).
  \label{eq:W}
\end{equation}
Nel caso in cui $f_i$ non sia multipla di $F$ si può comunque
approssimare l'ampiezza dei picchi con $\frac{A_iTN}{2}$.

Osservando il modulo della DFT $Y(kF)$ del segnale in ingresso,
riportata nella \fig{plot:y_dft}, si vede che i picchi relativi ai
rumori hanno ampiezze molto più grandi del resto dello spettro del
segnale, quindi per rilevarli automaticamente la funzione
\inlcd{find_noise} (\cod{lst:find_noise.m}) calcola la DFT di $y$
nell'intervallo $[14, 18]$ s, dove i tre rumori sono tutti presenti
per l'intera durata dell'intervallo, e cerca il $k_i$ per cui
l'ampiezza della trasformata è massima. Se
\begin{equation}
  |Y_{\mathit{last}}(k_iF)| \geq \mathit{min\_thresh} \cdot \frac{A_\mathit{min}TN}{2}
  \label{eq:find_noise}
\end{equation}
la funzione rileva un rumore a frequenza $f_i = k_iF$, altrimenti
ritorna \inlcd{NaN} per segnalare che non è stato rilevato alcun
rumore. Nella formula \eqn{eq:find_noise} $A_\mathit{min} = 0.1$ è la
minima ampiezza possibile per i disturbi, $TN = T_p = \SI{4}{\s}$ è la
durata dell'intervallo di osservazone di $y$ e $\mathit{min\_thresh} =
0.9$ abbassa del \SI{10}{\percent} la soglia minima per tenere conto
di eventuali componenti a frequenza $f_i$ del segnale originale
$x(nT)$ che potrebbero abbassare l'ampiezza dei picchi.

\begin{figure}[h]
  \centering
  \includegraphics[width=0.7\textwidth]{y_dft}
  \caption{Spettro del segnale di ingresso $y(t)$}
  \label{plot:y_dft}
\end{figure}

Nel grafico in \fig{plot:y_dft_w_noises} e nella
\tab{tab:noise_params} sono riportate le tre frequenze individuate.

\begin{figure}[h]
  \centering
  \includegraphics[width=0.7\textwidth]{y_dft_w_noises}
  \caption{Rumori rilevati}
  \label{plot:y_dft_w_noises}
\end{figure}

\begin{table}[h]
  \centering
  \begin{tabular}{lrrrr}
    Rumore & $f_i$                 & $A_i$ & $n_i$ & $n_iT$    \\
    \hline
    $w_1$ & \SI{2.2}{\kilo\hertz} & 0.1 & 144059 & \SI{6}{\s} \\
    $w_2$ & \SI{2.6}{\kilo\hertz} & 0.1 & 144128 & \SI{6}{\s} \\
    $w_3$ & \SI{5.3}{\kilo\hertz} & 0.2 & 216040 & \SI{9}{\s} \\
  \end{tabular}
  \caption{Parametri dei rumori}
  \label{tab:noise_params}
\end{table}

\subsection{Ampiezza}
Nota la frequenza di ciascuno dei rumori $f_i = k_iF$ si può usare
l'equazione \eqn{eq:W} per ricavare l'ampiezza $A_i$ del rumore $w_i$
dall'ampiezza dei picchi della DFT.  La funzione
\inlcd{find_noise_amplitude} (\cod{lst:find_noise_amplitude.m})
suppone trascurabile il contributo del segnale $x$ e calcola $A_i$
dalla trasformata $Y_\mathit{last}(kF)$ degli ultimi 4 secondi di $y$:
\begin{equation}
  |A_i| = \frac{2|Y_\mathit{last}(k_iF)|}{T_p}
\end{equation}
con $T_p = \SI{4}{\s}$. Le ampiezze dei tre rumori rilevati sono
riportate nella \tab{tab:noise_params} ed evidenziate, insieme agli
istanti di attacco, nel grafico in \fig{plot:y_w_noises}.

\begin{figure}[h]
  \centering
  \includegraphics[width=0.7\textwidth]{y_w_noise}
  \caption{Ampiezza e istante di attacco dei rumori}
  \label{plot:y_w_noises}
\end{figure}

\subsection{Istanti iniziali}
Per rilevare l'istante di attacco di ciascuno dei rumori $w_i$ è
possibile filtrare la componente a frequenza $f_i$ del segnale $y$ e
cercare il primo campione in cui questa è sufficientemente diversa da
zero. I filtri usati sono IIR di ordine 2 con funzione di
trasferimento
\begin{align}
  H_\mathit{sel} &= \frac{b_0}{
    (1 - re^{j\pi f_i/F_c}z^{-1})
    (1 - re^{-j\pi f_i/F_c}z^{-1})} \\
  &= \frac{2(1-r)|\sin(2\pi f_i/F_c)|}
  {1 - 2r\cos(2\pi f_i/F_c)z^{-1} + r^2z^{-2}}
\end{align}
dove $b_0 = 2(1-r)|\sin(2\pi f_i/F_c)|$ è scelto in modo che il
guadagno a frequenza $f_i$ sia circa 1. In \fig{plot:Hsel} è tracciata
la risposta in frequenza del filtro per $f_1$.

\begin{figure}[h]
  \centering
  \includegraphics[width=0.7\textwidth]{Hsel}
  \caption{Filtro selettivo per $f_1$}
  \label{plot:Hsel}
\end{figure}

La funzione \inlcd{find_noise_start} (\cod{lst:find_noise_start.m})
chiama \inlcd{single_freq_filter} (\cod{lst:single_freq_filter.m}) per
calcolare i coefficienti di un filtro selettivo centrato in $f_i$ con
fattore di merito $Q_\mathit{sel}$, filtra il segnale $y(nT)$ e poi
prende come istante iniziale del disturbo il primo campione che
raggiunge $\mathit{thresh}$ volte l'ampiezza massima di $y(nT)$ nei
primi 4 secondi. I parametri $\mathit{thresh} = 10$ e $Q_\mathit{sel}
= 400$ sono stati scelti per tentativi provando alcuni dei segnali
forniti.  Gli istanti di attacco stimati e i corrispondenti campioni
sono riportati nella \tab{tab:noise_params} e nella
\fig{plot:y_w_noises} insieme alle ampiezze dei rumori.

\subsection{Filtraggio}
Per attenuare un disturbo a frequenza $f_i$ viene usato un filtro
notch con funzione di trasferimento
\begin{equation}
  H_i(z) = \frac{(1 - e^{j2\pi f_i/F_c}z^{-1})(1 - e^{-j2\pi f_i/F_c}z^{-1})}
    {(1 - re^{j2\pi f_i/F_c}z^{-1})(1 - re^{-j2\pi f_i/F_c}z^{-1})}.
\end{equation}
Moltiplicando i due termini al numeratore e denominatore e applicando
la formula di Eulero si ottengono i coefficienti della funzione di
trasferimento:
\begin{equation}
  H_i(z) = \frac{1 - 2\cos(2\pi f_i/F_c)z^{-1} + z^{-2}}
  {1 - 2r\cos(2\pi f_i/F_c)z^{-1} + r^2z^{-2}}.
\end{equation}

La funzione \inlcd{notch_filter} (\cod{lst:notch_filter.m}) calcola e
restituisce i coefficienti di $H(z)$ a partire dalla frequenza da
cancellare $f_i$ e dal fattore di merito $Q = \frac{f_i}{\Delta f}$
desiderati. Se il parametro $r$ è circa 1 si può approssimare la banda
del filtro con $\Delta \theta_{\SI{3}{\dB}} \simeq 2\delta$ dove
$\delta = 1 - r$ è la distanza dei due poli dal cerchio di raggio
unitario e $\Delta \theta_{\SI{3}{\dB}} = \frac{2\pi}{F_c}\Delta
f_{\SI{3}{\dB}}$. La funzione usa questa approssimazione per scegliere
il valore di $r = 1 - \frac{2\pi}{F_c} \frac{f_i}{Q}$.

Nelle righe 35-36 del \cod{lst:progetto.m} il segnale di ingresso
viene filtrato una volta per ciascuna frequenza dei rumori e queste
operazioni sono equivalenti a filtrarlo una volta sola con il filtro
$H(z) = H_1(z)H_2(z)H_3(z)$ di cui è tracciata la risposta in
frequenza in \fig{plot:notch} e sono riportati i parametri nella
\tab{tab:notch}. Il valore di $Q = 10$ è stato scelto per tentativi:
valori troppo piccoli comportano una attenuazione eccessiva del
segnale nelle frequenze vicino alle $f_i$, mentre valori troppo grandi
comportano un lungo transitorio prima che i filtri cancellino il
rumore.

\begin{figure}[h]
  \centering
  \includegraphics[width=0.7\textwidth]{H}
  \caption{Cascata dei tre filtri notch usati}
  \label{plot:notch}
\end{figure}

\begin{table}[h]
  \centering
  \begin{tabular}{lrrrr}
    Filtro & $f_0$                 & $\Delta f_{\SI{3}{\dB}}$ & $r$    \\
    \hline
    $H_1$  & \SI{2.2}{\kilo\hertz} & \SI{220}{\hertz}     & 0.942 \\
    $H_2$  & \SI{2.6}{\kilo\hertz} & \SI{260}{\hertz}     & 0.932 \\
    $H_3$  & \SI{5.3}{\kilo\hertz} & \SI{530}{\hertz}     & 0.861 \\
  \end{tabular}
  \caption{Parametri dei filtri notch}
  \label{tab:notch}
\end{table}

Nel grafico di \fig{plot:y_dft_out} è tracciato lo spettro del segnale
dopo il filtraggio dove si vede che i picchi relativi ai tre rumori
sono stati cancellati. I tre rumori non sono più udibili ad eccezione
di un ``click'' in corrispondenza degli istanti di attacco dovuto alla
risposta transitoria dei filtri.

\begin{figure}[h]
  \centering
  \includegraphics[width=0.7\textwidth]{y_dft_out}
  \caption{Spettro del segnale filtrato}
  \label{plot:y_dft_out}
\end{figure}

\clearpage

\section{Codice Matlab}
\includecode{notch\string_filter.m}
\includecode{single\string_freq\string_filter.m}
\includecode{find\string_noise.m}
\includecode{find\string_noise\string_start.m}
\includecode{find\string_noise\string_amplitude.m}
\includecode{progetto.m}
\end{document}
